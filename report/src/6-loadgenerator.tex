\chapter{Load Generator}

%%%%%%%%%%%%%%%%%%%%%%%%%%%%%%%%%%%%%%%%%%%%%%%%%%
\section{K6 Operator}
Il generatore di carico K6 può essere installato localmente su una macchina Linux, MacOS o Windows, ma ha anche il grande vantaggio che può esere all'interno di un'infrestruttura cloud. Nel nostro progetto questo è utile e sensato se distruiamo K6 all'interno del cluster Minikube dove è in esecuzione la nostra applicazione; in questo contesto l'istanza K6 prende il nome di K6 Operator.

Per distribuire K6 in un cluster Kubernetes ci sono tre modi, ma quello che abbiamo usato è, come in precedenza, Helm. La \href{https://grafana.com/docs/k6/latest/set-up/set-up-distributed-k6/}{documentazione di K6} fornisce istruzioni molto precise su come gestire la configrazione e l'installazione. Nel nostro caso è stato parametrizzato tutto nel file \textit{values.yaml} che possiamo vedere nel Listing~\ref{lst:k6-values}.
\begin{lstlisting}[caption=K6 \textit{values.yaml} file, label=lst:k6-values]
targetUrl: "http://my-service:80/api/endpoint"
vus: 10
duration: "10s"
script: |
    import http from 'k6/http';
    import { sleep } from 'k6';
    export let options = {
        vus: __ENV.VUS,
        duration: __ENV.DURATION,
    };
    export default function () {
        http.get(__ENV.TARGET_URL);
        sleep(1);
    }
\end{lstlisting}
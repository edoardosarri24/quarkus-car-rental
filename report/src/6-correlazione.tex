\chapter{Correlazione}
Fino a questo momento i nostri microservizi non avevano nessun tipo di correlazione: il tempo di esecuzione del microserivizion $A$ è totalmente indipendente dal tempo di esecuzione di $B$, per qualunque $A$ e $B$. In questo capito andremo ad aggiungere correlazione tra qualche microservizio per cercare di dimostrare un aspetto tipico della tecnica di analisi composizionale di Eulero: siccome le distrubizioni (i.e., PDF e CDF) del tempo di esecuzione E2E sono costruite con un approccio bottom-up, cioè componendo le distribuzioni marginali fino ad arrivare al risultato, il modello perde tutte le correlazioni che ci possono essere; quando i tempi di esecuzione sono in un qualche modo correlati, allora il sistema di analisi dovrebbe funzionare peggio, cioè l'area tra le due curve sovrebbe aumentare.

%%%%%%%%%%%%%%%%%%%%%%%%%%%%%%%%%%%%%%%%%%%%%%%%%%
\section{Implementazione}
Vediamo come implementare la dipendenza tra microservizi in modo semplice ma funzionale. Andremo ad aggiungere due livelli: nella prima avremo una correlazione solamente tra due microservizi; nella seconda aggiungeremo alla precedente anche un altro tipo di correlaizone. Ci aspettimao che gli esperimenti sulla correlazione maggiore siano meno precisi.

\subsection{Bassa correlazione}

\subsection{Alta correlazione}
\chapter{Correlazione}
Fino a questo momento i nostri microservizi non avevano nessun tipo di correlazione: il tempo di esecuzione del microserivizion $A$ è totalmente indipendente dal tempo di esecuzione di $B$, per qualunque $A$ e $B$. In questo capito andremo ad aggiungere correlazione tra qualche microservizio per cercare di dimostrare un aspetto tipico della tecnica di analisi composizionale di Eulero: siccome le distrubizioni (i.e., PDF e CDF) del tempo di esecuzione E2E sono costruite con un approccio bottom-up, cioè componendo le distribuzioni marginali fino ad arrivare al risultato, il modello perde tutte le correlazioni che ci possono essere; quando i tempi di esecuzione sono in un qualche modo correlati, allora il sistema di analisi dovrebbe funzionare peggio, cioè l'area tra le due curve sovrebbe aumentare.

Per capire il livello di correlazione iniziale possiamo osservare la Figura~\ref{fig:correlation_no_correlation}: notiamo come tutti i servizi hanno una correlazione a due a due molto bassa (i.e., vicina a 0).

\begin{figure}[htbp]
    \centering
    \begin{subfigure}{0.3\textwidth}
        \centering
        \includegraphics[width=\textwidth]{images/7-correlazione/correlation0/correlation_matrix_first_choice.pdf}
        \caption{First-coice path}
    \end{subfigure}
    \hfill
    \begin{subfigure}{0.3\textwidth}
        \centering
        \includegraphics[width=\textwidth]{images/7-correlazione/correlation0/correlation_matrix_second_choice.pdf}
        \caption{Second-coice path}
    \end{subfigure}
    \hfill
    \begin{subfigure}{0.3\textwidth}
        \centering
        \includegraphics[width=\textwidth]{images/7-correlazione/correlation0/correlation_matrix_third_choice.pdf}
        \caption{Third-coice path}
    \end{subfigure}
    \caption{Correlazioni iniziali, i.e. senza l'aggiunta esplicita nel codice.}
    \label{fig:correlation_no_correlation}
\end{figure}

È importante osservare che il numero di tracce dei tre esperimenti, senza conrrelazione, con basse e con alta correlazione, sono eseguiti su un numero di tracce simile, circa 5500.

%%%%%%%%%%%%%%%%%%%%%%%%%%%%%%%%%%%%%%%%%%%%%%%%%%
\section{Implementazione}
Vediamo come implementare la dipendenza tra microservizi in modo semplice ma funzionale. Osserviamo che vogliamo aggiungere una correlazione tra i tempi di esecuzione di due microservizi; è importante non aggiungere una dipendenza funzionale, ma temporale tra una coppia.

Andremo ad aggiungere due livelli: nella prima avremo una correlazione solamente tra due microservizi; nella seconda aggiungeremo alla precedente anche un altro tipo di correlaizone. Ci aspettimao che gli esperimenti sulla correlazione maggiore siano meno precisi.

\subsection{Bassa correlazione}
In questo esperimento si è aggiunta una sola dipendenza, in particolare tra \textit{users-service} e \textit{reservation-service}.

Prima di chiamare \textit{reservation-service/reservation}, il microservizio \textit{users-service} campiona un tempo di esecuzione $t$, in cui fa una busy-wait, da una distribuzione esponenziale con tasso $\tfrac{1}{7}$; il servizio \textit{reservation} farà una busy wait per un tempo di dipendente da quanto campionato precedentemente da \textit{users-service}, secondo la funzoone $f(t)=\begin{cases}5ms & \text{ se } t<7\\15ms & \text{altrimenti}\end{cases}$. L'implementazione avviene passando un parametro booleano nella chiamata di \textit{users-service} a \textit{reservation-service}.

La matrice di correlazione diventa, rispetto alla precedente, quella mostrata in Figura~\ref{fig:correlation_low_correlation}.

\begin{figure}[htbp]
    \centering
    \begin{subfigure}{0.3\textwidth}
        \centering
        \includegraphics[width=\textwidth]{images/7-correlazione/low_correlation/correlation_matrix_first_choice.pdf}
        \caption{First-coice path}
    \end{subfigure}
    \hfill
    \begin{subfigure}{0.3\textwidth}
        \centering
        \includegraphics[width=\textwidth]{images/7-correlazione/low_correlation/correlation_matrix_second_choice.pdf}
        \caption{Second-coice path}
    \end{subfigure}
    \hfill
    \begin{subfigure}{0.3\textwidth}
        \centering
        \includegraphics[width=\textwidth]{images/7-correlazione/low_correlation/correlation_matrix_third_choice.pdf}
        \caption{Third-coice path}
    \end{subfigure}
    \caption{Matrici di correlazione nei servizi con bassa correlazione.}
    \label{fig:correlation_low_correlation}
\end{figure}

\subsection{Alta correlazione}
L'implementazione con alta correlazione è definita sulla basa di quella precedente a bassa correlazione, a cui è stata aggiunta un'ulteriore dipendenza tra i tempi di esecuzione di microservizi.

In particoalre dopo che i microservizi paralleli \textit{first-parallel-service} e \textit{second-parallel-service} ritornano, il chiamante \textit{start-parallel-service} esegue per un tempo che è definito dalla somma dei tempi di esecuzione dei due servizi paralleli. Questo viene implementato facendo ritornare ai due servizi un double che rappresenta il tempo per essi hanno eseguit la loro busywait; \textit{start-parallel-service} eseguirà una busy-wait per questo tempo.

La matrice di correlazione in questo caso è mostrata Figura~\ref{fig:correlation_high_correlation}.

\begin{figure}[htbp]
    \centering
    \begin{subfigure}{0.3\textwidth}
        \centering
        \includegraphics[width=\textwidth]{images/7-correlazione/high_correlation/correlation_matrix_first_choice.pdf}
        \caption{First-coice path}
    \end{subfigure}
    \hfill
    \begin{subfigure}{0.3\textwidth}
        \centering
        \includegraphics[width=\textwidth]{images/7-correlazione/high_correlation/correlation_matrix_second_choice.pdf}
        \caption{Second-coice path}
    \end{subfigure}
    \hfill
    \begin{subfigure}{0.3\textwidth}
        \centering
        \includegraphics[width=\textwidth]{images/7-correlazione/high_correlation/correlation_matrix_third_choice.pdf}
        \caption{Third-coice path}
    \end{subfigure}
    \caption{Matrici di correlazione nei servizi con alta correlazione.}
    \label{fig:correlation_high_correlation}
\end{figure}

%%%%%%%%%%%%%%%%%%%%%%%%%%%%%%%%%%%%%%%%%%%%%%%%%%
\section{Risultati}
Vediamo adesso i risultati ottenuti con le due simulazioni, cioè i grafici delle distribuzioni reali e ottenuti con Eulero.
\begin{itemize}
    \item I risutati ottenuti per la bassa correlazione sono mostrati in Figura~\ref{fig:distribution_low_correlation}.
    \item I risutati ottenuti per l'alta correlazione sono mostrati in Figura~\ref{fig:distribution_high_correlation}.
\end{itemize}

\begin{figure}[htbp]
    \centering
    \begin{subfigure}{\textwidth}
        \centering
        \includegraphics[width=\textwidth]{images/7-correlazione/compare_e2e_distribution_low_correlation.pdf}
        \caption{Bassa correlazione.}
        \label{fig:distribution_low_correlation}
    \end{subfigure}
    \vfill
    \begin{subfigure}{\textwidth}
        \centering
        \includegraphics[width=\textwidth]{images/7-correlazione/compare_e2e_distribution_high_correlation.pdf}
        \caption{Alta correlazione.}
        \label{fig:distribution_high_correlation}
    \end{subfigure}
    \caption{Distribuzioni negli esperimenti con correlazione diversa.}
\end{figure}

Possiamo notare come l'area delle curve aumenta all'aumentare della correlazione, segno che l'analisi composizionale eseguita da Eulero non riesce a catturare tali dipendenze.

%%%%%%%%%%%%%%%%%%%%%%%%%%%%%%%%%%%%%%%%%%%%%%%%%%
\section{Bande di confidenza}
Analizziamo adesso esattamente le stesse bande di confidenza viste nella Sezione~\ref{sec:confidence_bands}.
\begin{itemize}
    \item Per le distribuzioni con bassa correlazione possiamo osservere la Figura~\ref{fig:confidence_bands_low_correlation}.
    \item Per le distribuzioni con alta correlazione possiamo osservere la Figura~\ref{fig:confidence_bands_high_correlation}.
\end{itemize}


\begin{figure}[htbp]
    \centering
    \begin{minipage}{0.45\textwidth}
        \centering
        \begin{subfigure}{\textwidth}
            \centering
            \includegraphics[width=\textwidth]{images/7-correlazione/low_correlation/confidence_bands/compare_e2e_distribution_cp.pdf}
            \caption{Pointwise con Clopper-Pearson.}
        \end{subfigure}
        \vfill
        \begin{subfigure}{\textwidth}
            \centering
            \includegraphics[width=\textwidth]{images/7-correlazione/low_correlation/confidence_bands/compare_e2e_distribution_dkw.pdf}
            \caption{Simultaneus con DKW}
        \end{subfigure}
    \end{minipage}
    \hfill
    \begin{minipage}{0.45\textwidth}
        \centering
        \begin{subfigure}{\textwidth}
            \centering
            \includegraphics[width=\textwidth]{images/7-correlazione/low_correlation/confidence_bands/compare_e2e_distribution_bootstrap_pointwise.pdf}
            \caption{Pointwise con bootstrapping.}
        \end{subfigure}
        \vfill
        \begin{subfigure}{\textwidth}
            \centering
            \includegraphics[width=\textwidth]{images/7-correlazione/low_correlation/confidence_bands/compare_e2e_distribution_bootstrap_simultaneous.pdf}
            \caption{Simultaneus con bootstrapping.}
        \end{subfigure}
    \end{minipage}
    \caption{Bande di confidenza per le distribuzioni con bassa correlazione.}
    \label{fig:confidence_bands_low_correlation}
\end{figure}

\begin{figure}[htbp]
    \centering
    \begin{minipage}{0.45\textwidth}
        \centering
        \begin{subfigure}{\textwidth}
            \centering
            \includegraphics[width=\textwidth]{images/7-correlazione/high_correlation/confidence_bands/compare_e2e_distribution_cp.pdf}
            \caption{Pointwise con Clopper-Pearson.}
        \end{subfigure}
        \vfill
        \begin{subfigure}{\textwidth}
            \centering
            \includegraphics[width=\textwidth]{images/7-correlazione/high_correlation/confidence_bands/compare_e2e_distribution_dkw.pdf}
            \caption{Simultaneus con DKW}
        \end{subfigure}
    \end{minipage}
    \hfill
    \begin{minipage}{0.45\textwidth}
        \centering
        \begin{subfigure}{\textwidth}
            \centering
            \includegraphics[width=\textwidth]{images/7-correlazione/high_correlation/confidence_bands/compare_e2e_distribution_bootstrap_pointwise.pdf}
            \caption{Pointwise con bootstrapping.}
        \end{subfigure}
        \vfill
        \begin{subfigure}{\textwidth}
            \centering
            \includegraphics[width=\textwidth]{images/7-correlazione/high_correlation/confidence_bands/compare_e2e_distribution_bootstrap_simultaneous.pdf}
            \caption{Simultaneus con bootstrapping.}
        \end{subfigure}
    \end{minipage}
    \caption{Bande di confidenza per le distribuzioni con alta correlazione.}
    \label{fig:confidence_bands_high_correlation}
\end{figure}
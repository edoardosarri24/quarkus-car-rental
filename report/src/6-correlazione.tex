\chapter{Correlazione}
Fino a questo momento i nostri microservizi non avevano nessun tipo di correlazione: il tempo di esecuzione del microserivizion $A$ è totalmente indipendente dal tempo di esecuzione di $B$, per qualunque $A$ e $B$. In questo capito andremo ad aggiungere correlazione tra qualche microservizio per cercare di dimostrare un aspetto tipico della tecnica di analisi composizionale di Eulero: siccome le distrubizioni (i.e., PDF e CDF) del tempo di esecuzione E2E sono costruite con un approccio bottom-up, cioè componendo le distribuzioni marginali fino ad arrivare al risultato, il modello perde tutte le correlazioni che ci possono essere; quando i tempi di esecuzione sono in un qualche modo correlati, allora il sistema di analisi dovrebbe funzionare peggio, cioè l'area tra le due curve sovrebbe aumentare.

Per capire il livello di correlazione iniziale possiamo osservare la Figura~\ref{fig:correlation_no_correlation}: notiamo come tutti i servizi hanno una correlazione a due a due molto bassa (i.e., vicina a 0).

\begin{figure}[htbp]
    \centering
    \begin{subfigure}{0.3\textwidth}
        \centering
        \includegraphics[width=\textwidth]{images/7-correlazione/correlation0/correlation_matrix_first_choice.pdf}
        \caption{First-coice path}
    \end{subfigure}
    \hfill
    \begin{subfigure}{0.3\textwidth}
        \centering
        \includegraphics[width=\textwidth]{images/7-correlazione/correlation0/correlation_matrix_second_choice.pdf}
        \caption{Second-coice path}
    \end{subfigure}
    \hfill
    \begin{subfigure}{0.3\textwidth}
        \centering
        \includegraphics[width=\textwidth]{images/7-correlazione/correlation0/correlation_matrix_third_choice.pdf}
        \caption{Third-coice path}
    \end{subfigure}
    \caption{Correlazioni iniziali, i.e. senza l'aggiunta esplicita nel codice.}
    \label{fig:correlation_no_correlation}
\end{figure}

%%%%%%%%%%%%%%%%%%%%%%%%%%%%%%%%%%%%%%%%%%%%%%%%%%
\section{Implementazione}
Vediamo come implementare la dipendenza tra microservizi in modo semplice ma funzionale. Osserviamo che vogliamo aggiungere una correlazione tra i tempi di esecuzione di due microservizi; è importante non aggiungere una dipendenza funzionale, ma temporale tra una coppia.

Andremo ad aggiungere due livelli: nella prima avremo una correlazione solamente tra due microservizi; nella seconda aggiungeremo alla precedente anche un altro tipo di correlaizone. Ci aspettimao che gli esperimenti sulla correlazione maggiore siano meno precisi.

\subsection{Bassa correlazione}
In questo esperimento si è aggiunta una sola dipendenza, in particolare tra \textit{users-service} e \textit{reservation-service}.

Prima di chiamare \textit{reservation/reservation}, \textit{users-service} campiona un tempo di esecuzione, in cui fa una busy-wait, da una distribuzione esponenziale con tasso $\tfrac{1}{7}$; se il tempo campionato è minore di 7 (i.e., la media) allora passa a \textit{reservation-service} il valore \texttt{true}, altrimenti \texttt{false}. A questo punto \textit{reservation-service} esegue una busy wait per 5ms se il valore è \texttt{true}, per 15ms altrimenti.

La matrice di correlazione diventa, rispetto alla precedente, quella mostrata in Figura~\ref{fig:correlation_low_correlation}.

\begin{figure}[htbp]
    \centering
    \begin{subfigure}{0.3\textwidth}
        \centering
        \includegraphics[width=\textwidth]{images/7-correlazione/low_correlation/correlation_matrix_first_choice.pdf}
        \caption{First-coice path}
    \end{subfigure}
    \hfill
    \begin{subfigure}{0.3\textwidth}
        \centering
        \includegraphics[width=\textwidth]{images/7-correlazione/low_correlation/correlation_matrix_second_choice.pdf}
        \caption{Second-coice path}
    \end{subfigure}
    \hfill
    \begin{subfigure}{0.3\textwidth}
        \centering
        \includegraphics[width=\textwidth]{images/7-correlazione/low_correlation/correlation_matrix_third_choice.pdf}
        \caption{Third-coice path}
    \end{subfigure}
    \caption{Matrici di correlazione nei servizi con bassa correlazione.}
    \label{fig:correlation_low_correlation}
\end{figure}

\subsection{Alta correlazione}
L'implementazione con alta correlazione è definita sulla basa di quella precedente a bassa correlazione, a cui è stata aggiunta un'ulteriore dipendenza tra i tempi di esecuzione di microservizi.

In particoalre TODO.

%%%%%%%%%%%%%%%%%%%%%%%%%%%%%%%%%%%%%%%%%%%%%%%%%%
\section{Risultati}
Vediamo adesso i risultati ottenuti con le due simulazioni, cioè i grafici delle distribuzioni reali e ottenuti con Eulero.
\begin{itemize}
    \item I risutati ottenuti per la bassa correlazione sono mostrati in Figura~\ref{fig:distribution_low_correlation}.
    \item I risutati ottenuti per l'alta correlazione sono mostrati in Figura~\ref{fig:distribution_high_correlation}.
\end{itemize}

\begin{figure}[htbp]
    \centering
    \begin{subfigure}{\textwidth}
        \centering
        \includegraphics[width=\textwidth]{images/7-correlazione/compare_e2e_distribution_low_correlation.pdf}
        \caption{Bassa correlazione}
        \label{fig:distribution_low_correlation}
    \end{subfigure}
    \hfill
    \begin{subfigure}{\textwidth}
        \centering
        \includegraphics[width=\textwidth]{images/7-correlazione/compare_e2e_distribution_low_correlation.pdf}
        \caption{Second-coice path}
        \label{fig:distribution_high_correlation}
    \end{subfigure}
    \caption{Distribuzioni negli esperimenti con correlazione diversa.}
\end{figure}
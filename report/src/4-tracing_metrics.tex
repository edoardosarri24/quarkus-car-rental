\chapter{Tracing and Metrics}
Dopo aver fatto il deployment completo della nostra applicazione, abbiamo aggiunto il tracciamento delle chiamate che, a partire da un servizio, vengono propagate attraverso altri per completare una funzionalità.

%%%%%%%%%%%%%%%%%%%%%%%%%%%%%%%%%%%%%%%%%%%%%%%%%%
\section{Tracing}
In applicazioni con migliaia di micro servizi, seguire la cascata di chiamate o valutare in che punto si è verificato un errore è molto complesso. Tramite il tracing non solo possiamo capire quale micro servizio viene chiamato e da chi, ma possiamo anche analizzare i tempi con cui ogni richiesta è stata servita, sia E2E che all'interno del singolo servizio.

\myskip

Il funzionamento solitamente è abbastanza semplice. Durante le varie chiamate viene passato anche un ID (univoco) della traccia. Il trasporto di questo ID avviene tramite una qualche funzionalità del protocollo di trasporto: in HTTP viene usato il relativo header.

\subsection{OpenTelemetry}
OpenTelemetry, noto anche come OTel, è un insieme di tool e API che permette di collezionare ed esportare le telemetrie di un'applicazione a micro servizi. Oltre alle telemetrie in realtà possono anche essere gestite metriche e logs, ma in questo campo non è stabile e si preferisce usare altro (e.g., MicroMeter).

\myskip

Fornisce un protocollo, detto OTLP, che permette di esportare le telemetrie dalle applicazioni verso il tool OpenTelemetry Collector. A quest'ultimo si possono collegare vari back end (e.g., Jaeger) per la loro visualizzazione.

Usare il Collector non è necessario: si possono inviare i dati direttamente al backend: in piccole applicazioni è forse una scelta migliore per la facilità di manutenzione, ma se l'applicazione deve scalare allora si hanno numero vantaggi usando il Collector, come il filtraggio, il batching e la riprova in caso di un qualche fallimento.

\myskip

Per utilizzare OTel nei vari micro servizi della nostra applicazione si deve semplicemente aggiungere l'estensione \textit{quarkus-opentelemetry} nel pom. In questo modo Quarkus inizia a collezionare telemetrie in automatico.

\subsection{Jaeger}
Per utilizzare \href{https://www.jaegertracing.io}{Jaeger} nei vari microservizi, e quindi avere la possibilità di visualizzare la cascata di chiamate, si deve aggiungere delle configurazioni nel file \textit{application.properties}. Queste configurazioni sono abbastanza standard e non dipendono molto dal microservizio specifico, se non per il nome.

Per vedere un esempio possiamo visualizzare le righe di configurazioni aggiunte per \textit{users-service} all'interno del Listing~\ref{listing:users-jaeger}.
\begin{lstlisting}[caption=Jaeger configuration for \textit{users-service}, label=listing:users-jaeger]
# jaeger
quarkus.kubernetes.env.vars.otel-service-name=users-service
quarkus.otel.resource.attributes=service.name=users-service
quarkus.kubernetes.env.vars.otel-exporter-otlp-endpoint=http://jaeger-collector:4317
quarkus.otel.exporter.otlp.endpoint=http://jaeger-collector:4317
quarkus.otel.traces.sampler=always_on
quarkus.log.console.format=%d{HH:mm:ss} %-5p traceId=%X{traceId}, parentId=%X{parentId}, spanId=%X{spanId}, sampled=%X{sampled} [%c{2.}] (%t) %s%e%n
\end{lstlisting}
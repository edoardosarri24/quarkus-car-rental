\section{Tracing}

\subsection{Software Architecture and Methodologies \& Quantitative Evaluation of Stochastic Models}

\begin{frame}{Tracing}
    \begin{block}{Tracing precedente}
        \begin{itemize}
            \item I microservizi ricevono richieste.
            \item OpenTelemetry raccoglie i dati dai microservizi. In Quarkus questo è semplice tramite l'estensione \textit{quarkus-opentelemetry}.
            \item OTel invia al backend (e.g., Jaeger). Serve configurazione in \textit{application.properties}.
        \end{itemize}
    \end{block}
    \centering
    \includegraphics[width=.5\textwidth]{images/7-tracing/old_otel_workflow/old_otel_workflow.pdf}
\end{frame}

\begin{frame}{Tracing}
    \begin{block}{Tracing attuale - Flusso}
        \begin{itemize}
            \item I microservizi ricevono richieste.
            \item OpenTelemetry raccoglie i dati dai microservizi. In Quarkus questo è semplice tramite l'estensione \textit{quarkus-opentelemetry}.
            \item OTel invia le statistiche all'Otel-collector, dove possiamo eseguire operazioni intermedie (e.g., filtraggio, aggregazione).
            \item Il collector invia i dati a vari back-end.
        \end{itemize}
    \end{block}
    \centering
    \includegraphics[width=.7\textwidth]{images/7-tracing/new_otel_workflow.pdf}
\end{frame}

\begin{frame}{Tracing}
    \begin{block}{Tracing attuale - Conseguenze}
        \begin{itemize}
            \item Si aggiuge complessità, soprattutto rispetto alla versione con la sola estensione Quarkus.
            \item Per progetti semplici (o all'inizio di uno complesso) è meglio semplificare.
            \item Per progetti complessi, che devono scalare, o se è necessario eseguire operazioni intermedie, è necessario il collector.
        \end{itemize}
    \end{block}
\end{frame}
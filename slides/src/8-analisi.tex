\section{Analisi E2E}

\subsection{Software Architecture and Methodologies \& Quantitative Evaluation of Stochastic Models}

\begin{frame}{Analisi E2E}
    \begin{block}{Obiettivo}
        Derivare una distribuzione (CDF) del tempo E2E del workflow in modo data-driven.
    \end{block}
    \begin{block}{Scenario}
        \begin{itemize}
            \item Caso medio: Si usano le statistiche descrittive (i.e., media e varianza).
            \item Soft-real time: per garantire un dato Service Level Agreements (SLA), si deve lavorare su un upper-bound del tempo di esecuzione.
        \end{itemize}
    \end{block}
\end{frame}

\begin{frame}{Analisi E2E}
    \begin{block}{Possibilità di analisi}
        \begin{itemize}
            \item Esatta: Fattibile per workflow non complessi e/o tempi di esecuzione Markoviani. In altri casi lo spazio degli stati esplode.
            \item Composizionale: Nello stato $i$-esimo si elimina il condizionamento sul tempo di esecuzione dello stato $i-1$-esimo. Si analizza in modo indipendente ogni blocco e si ricostruisce l'analisi E2E in modo bottom-up.
        \end{itemize}
    \end{block}
    \begin{block}{Dati necessari}
        \begin{itemize}
            \item \textbf{Workflow} delle chiamate.
            \item \textbf{Distribuzioni marginali} dei tempi di esecuzione.
        \end{itemize}
    \end{block}
\end{frame}

\begin{frame}{Analisi E2E}
    \begin{block}{Workflow}
        \centering
        \includegraphics[width=\textwidth]{images/6-workflow/new_workflow_jaeger.pdf}
        \includegraphics[width=\textwidth]{images/8-analisi/STPN_workflow.pdf}
    \end{block}
\end{frame}

\begin{frame}{Analisi E2E}
    \begin{block}{Distribuzioni}
        L'obiettivo è derivare la distribuzione che meglio approssima il tempo di esecuzione di ogni microservizio. Serve:
        \begin{itemize}
            \item Generare molte tracce.
            \item Ottenere statistiche descrittive dei tempi di esecuzione.
            \item Fitting delle distribuzioni su tali dati.
        \end{itemize}
    \end{block}
    \begin{block}{Generazione}
        La generazione è stata eseguita con K6.
        \begin{itemize}
            \item Job rilasciato con Helm.
            \item Ciclo di 30 secondi che chiama iterativamente \textit{users/reserve}.
        \end{itemize}
    \end{block}
\end{frame}

\begin{frame}{Analisi E2E}
    \begin{block}{Statistiche}
        A partire dalle tracce analizzate nel PVC collegato al pod \textit{analyzer}.
        \centering
        \includegraphics[width=\textwidth]{images/8-analisi/statistics.pdf}
    \end{block}
\end{frame}

\begin{frame}{Analisi E2E}
    \begin{block}{Statistiche}
        Si sono usato gli approssimanti di Whitt:
        \begin{itemize}
            \item $C_V=1$: Exp.
            \item $C_V>1$: HyperExp, XOR di due esponenziali.
            \item $\tfrac{1}{\sqrt{2}}\le C_V<1$: HypoExp: AND di due esponenziali.
        \end{itemize}
        \centering
        \includegraphics[width=\textwidth]{images/8-analisi/distributions.pdf}
    \end{block}
\end{frame}

\begin{frame}{Analisi E2E}
    \begin{block}{Distribuzione E2E (Eulero)}
        Di seguito la CDF e PDF del tempo di esecuzione E2E.
        \centering
        \includegraphics[width=.8\textwidth]{images/8-analisi/approx_distribution_plot.pdf}
    \end{block}
\end{frame}

\begin{frame}{Analisi E2E}
    \begin{block}{Distribuzione E2E reali vs approssimate}
        \begin{itemize}
            \item Questo se nono conoscessimo la vera distribuzione del tempo di esevuzione E2E.
            \item Confrontando la reale distribuzione con quella approssimata:
        \end{itemize}
        \centering
        \includegraphics[width=.85\textwidth]{images/8-analisi/E2E distribution function.pdf}
    \end{block}
\end{frame}

\begin{frame}{Analisi E2E}
    \begin{block}{Distribuzione E2E reali vs approssimate}
        \begin{itemize}
            \item Mettendo nel modello la reale distribuzione (i.e., Exp).
        \end{itemize}
        \centering
        \includegraphics[width=.85\textwidth]{images/8-analisi/E2E distribution function_with_exponential.pdf}
    \end{block}
\end{frame}

\begin{frame}{Analisi E2E}
    \begin{block}{Distribuzione E2E reali vs approssimate - ok}
        \begin{columns}[t]
            \begin{column}{.5\textwidth}
                \centering
                \includegraphics[width=.9\textwidth]{images/8-analisi/real_first-parallel-service_execTime.pdf} \\
                \vfill
                \includegraphics[width=.9\textwidth]{images/8-analisi/real_reservation-service-reservation-all_execTime.pdf}
            \end{column}
            \begin{column}{.5\textwidth}
                \centering
                \includegraphics[width=.9\textwidth]{images/8-analisi/real_second-choice-service_execTime.pdf} \\
                \vfill
                \includegraphics[width=.9\textwidth]{images/8-analisi/real_users-service-startParallel_execTime.pdf}
            \end{column}
        \end{columns}
    \end{block}
\end{frame}

\begin{frame}{Analisi E2E}
    \begin{block}{Distribuzione E2E reali vs approssimate - ko}
        \centering
        \includegraphics[width=.6\textwidth]{images/8-analisi/real_users-service-reserve_execTime.pdf}
        \vfill
        \centering
        \includegraphics[width=.6\textwidth]{images/8-analisi/real_start-parallel-service_execTime.pdf}
    \end{block}
\end{frame}